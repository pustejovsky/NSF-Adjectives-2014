\documentclass[11pt]{article}
\usepackage{graphicx}
\usepackage{amssymb}
\usepackage{epstopdf}
\DeclareGraphicsRule{.tif}{png}{.png}{`convert #1 `dirname #1`/`basename #1 .tif`.png}

\textwidth = 6.5 in
\textheight = 9 in
\oddsidemargin = 0.0 in
\evensidemargin = 0.0 in
\topmargin = 0.0 in
\headheight = 0.0 in
\headsep = 0.0 in



%\parskip = 0.2in
\parindent = 0.0in

\newtheorem{theorem}{Theorem}
\newtheorem{corollary}[theorem]{Corollary}
\newtheorem{definition}{Definition}

\begin{document}
\begin{center}
{\large {\bf Budget Justification for Brandeis University}}
\end{center}

 \vspace{-1.0em}
\subsection*{1. Personnel}


\begin{itemize}

\item The Principal Investigator of this proposal, James Pustejovsky, holds the TJX/Feldberg Chair in Computer Science at Brandeis University, where he is Full Professor and Chair of the Language and Linguistics Program. The PI is committed for the following time: one summer salary month in each of the three years.  Dr. Pustejovsky will be responsible for   development of Context-dependent Inferences (CDIs) relating to intensional adjectives and overall management of the project. Dr. Marc Verhagen will act as Project Manager at Brandeis, directing the graduate student in corpus collection, annotation work, coordination of AMT crowdsourced annotation, and analysis of the inter-annotation scoring. He will also assist in coordinating the integration of work from consultants Dr. Zaenen and Dr. Karttunen. 

 
\item One graduate student from the Computer Science Department is funded at full-time for the first year, and then at 9 months time for years two and three.  Her responsibilities will include writing the annotation guidelines, management of the undergraduate annotators, coordination of experiments at AMT, and 
 evaluation of the annotation results. 
 
\item Funds are requested for two Undergraduate students for six  calendar months for the first year, and for one student  during the second year. They will work as annotators and for AMT evaluation; they will also be used for programming support in the evaluation phase in the third year of the project.  

\end{itemize}

\subsection*{2.  Amazon Mechanical Turk Services}
\$7,000 is requested for Amazon Web Services  Mechanical Turking expenses for each of the first two years, and \$5,000 for the third year.  

 
\vspace{-1.0em}
\subsection*{3.  Travel}

\$6,000 is budgeted for travel for each year, for the following activities:
\vspace{-1.0em}
\begin{enumerate}
 
\item Travel to PI Meetings for each year. We have budgeted \$500 per year for the PI's travel,  accommodations, and meals. 
 
\item Travel to attend both national and international conferences, in order to report on findings from the proposed work. 

\end{enumerate}


\subsection*{Fringe}

For fiscal year 2015 and beyond, fringe benefits are computed at 29.7\% for the PI (Full-time faculty) and research staff, and 7.7\% for the graduate student.



\subsection*{Overhead}

Brandeis University's Indirect Costs: Modified total direct costs based on DHHS negotiated rates of June 19, 2013 as follows:
\begin{itemize}
\item July 1, 2015 -- June 30 2016: 62.5\% predetermined
\item July 1, 2016 and thereafter:  62.5\% provisional
\end{itemize}




 \end{document}

